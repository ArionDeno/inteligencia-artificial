%   Universidade Tuiuti Do Paraná      
%   Ciencia da Computação
%   Inteligencia Artficial
%   Professor: CHAUA COLUENE QUEIROLO BARBOSA DA SILVA
%   Aluno: Arion Denovaro
%-------------------------------------------------------
% 

% tipo de documento artigo
\documentclass[12pt,a4paper,twocolumn]{article}

%Blibiotecas 
\usepackage[brazil]{babel}
\usepackage[utf8]{inputenc}
\usepackage[T1]{fontenc}
\usepackage[caption=false]{subfig}
\usepackage{graphicx} %use de imgs
\usepackage{natbib}
\usepackage{hyperref}

\title{Analise de Estrategias de busca aplicada a Inteligencia Artificial}
\date{30/09/2020}
\author{Arion Denovaro  \\ \texttt  arion.barberi@utp.edu.br}

%-------incio-do-artigo-------------------
\begin{document}
\maketitle
% cria uma pagina em branco 
\newpage  
%descrevendo a resolução,
%por meio de estratégias de busca, 


%-------------------
\section{Introdução}
%-----------------------------
  	Desde tempos remotos sempre surgiu problemas na humanidade, inicialmente sendo na questão de sobrevivência, seja como: Alimentação, Abrigo, Saúde.	

	Com o passar  de eras, o homem conseguiu domar a natureza furiosa através da compreensão de seus mistérios, pode-se revelar um fantástico mundo oculto, desconhecido a ser descoberto, com doses de engenhosos artifícios simples, a natureza nos revelou que fazemos parte dela de uma forma mais sistemática e cultural, quanto gostaríamos.

	Mas passamos a interferir de tal modo que estamos causando mais problemas à natureza que nossos ancestrais, e com isso causando secas, desequilíbrios ambientais, sociais e doenças. 

	Não podemos nos distanciar da natureza e esquecer que somos parte dela, e por isso dependemos dela. Isso nos fez aprender e aplicar tais conhecimentos para desenvolvimento daquilo que chamamos de ciência, com a finalidade de melhorar nossas chances de sobrevivência.


%###################

%-------------------
\section{Fundamentação Téorica}
%-----------------------------
A inteligência artificial é uma área da computação que começou a surgir já no incio da informática, com o objetivo de tentar tornar o computador mais inteligente, poder tomar decisões mais eficientes para que com isso possa resolver problemas que até então não poderiam ser resolvidos, como: achar a saída de um labirinto, interpretar a fala humana, identificar objetos, interagir e responder humanos, entre outros.

 
%###################

%-------------------
\section{Cubo de Rubik}
%-------------------

%_________
\subsection{Introdução}
Cubo mágico, é um quebra-cabeça tridimensional, que procura mexer as faces do cubo de 3x3 (ou mais), de modo que cada face fique ordenado com uma única cor, por esse motivo o cubo mágico é um problema de ordenação auxiliado por uma de busca uniforme, uma IA conseguiria resolver o mesmo. Pois
deve-se buscar as posições e os dados, tanto largura como em profundidade., fazendo buscas para validar e verificar se a posição, ou os movimentos leva a 
a colocação da cor no quadrante adequado. Para isso deve-se esgotar as possibilidades e para isso a busca uniforme se encaixa adequadamente.


%-----------------------------------
\begin{figure}[h]
   \centering
   \includegraphics[width=4cm]{cubo1.jpg}
   \caption{Figura Cubo magico}
   \label{fig:cubo1} 
\end{figure}
%----------------------------------------
	
%_________
\subsection{Estado Inicial}
O Cubo estará desorganizado, com as cores embaralhadas, de forma aleatória.

%-----------------------------------
\begin{figure}[h]
   \centering
   \includegraphics[width=4cm]{cubo2.jpg}
   \caption{Figura Cubo magico embaralhado}
   \label{fig:cubo2} 
\end{figure}
%----------------------------------------

%_________
\subsection{Ações}
Rotacionar linhas das faces, na horizontal, vertical, virar o cubo.
%_________
\subsection{Modelo de Transição}
  \textbf{rotacionar as faces}:Girar qualquer uma das linhas externas, tanto na horizontal, vertical, em 90 graus, tanto rotação horaria quanto anti-horaria.



%-----------------------------------
\begin{figure}[h]
   \centering
   \includegraphics[width=4cm]{cubo3.jpg}
   \caption{Figura Cubo magico ações}
   \label{fig:cubo3} 
\end{figure}
%----------------------------------------

 \textbf{virar cubo}:  Virar o cubo em qualquer ângulo de 360 graus, normalmente se matem uma face com ângulo de 90 graus para Y-, para se manter a orientação visual, dispondo-se de modo que fique 3 faces visíveis.
%_________
\subsection{Custo de Caminho}
A menor quantidade de movimentos para resolver o cubo é de 28 movimentos aproximadamente.

%_________
\subsection{Teste do Objetivo}
As faces da cor que mudou de posição ao final, pertence a mesma cor da face que fica no meio?
Se Sim, movimento válido, caso contrário será movimento desnecessário.


%###################

%-------------------
\section{Missionários e Canibais}
%-------------------
%_________
\subsection{Introdução}
O problema dos missionários e canibais é um problema de ordenação, onde 3 canibais e 3 missionários estão viajando, para atravessar um rio, possuem de uma jangada que suporta somente no máximo 2 ocupantes por vez, não se pode exceder a quantidade de canibais em ambas as margens ou os missionários estarão em risco. Deve-se então organizar a travessia de modo que o número de canibais sempre seja menor ou igual aos números missionários, tal problema está classificado como de busca em largura, pois deve-se analisar os nos próximos, para estar em harmonia. Tal problema pode-se ser solucionado por busca em largura, que visa analisar a situação de forma de uma fila, verificando para ver quais escolhas são as mais adequadas. 


%_________
\subsection{Estado Inicial}
   Em uma das margens do rio, estão 3 missionários e 3 canibais, e o barco para no máximo 2 pessoas, do outro lado do rio está vazio a margem.

%_________
\subsection{Ações}
Colocar ocupantes no barco, mover barco para outra margem.


%_________
\subsection{Modelo de Transição}
 \textbf{colocar uma ou duas pessoas no barco}: pode-se escolher para colocar no barco qualquer um tanto faz ser missionário ou canibal, lembrando que o barco não pode ficar vazio, pois não opera de forma autônoma.



\textbf{colocar uma ou duas pessoas na margem}: permite-se colocar uma ou duas pessoas na margem em que o barco está.


\textbf{travesia do barco}:O barco atravessa para a margem oposta que está.

%_________
\subsection{Custo de Caminho}
O menor número de movimentos para resolver o problema é de 11 travessias.

%_________
\subsection{Teste do Objetivo}
 Quando todos os 6 membros da estiverem na outra margem.


%###################


%-------------------
\section{Problema das $n$ rainhas}
%-------------------
%_________
\subsection{Introdução}
O problema das $n$ rainhas, é um problema que pode ser usado para se representar conflito de processos em computação paralela, onde as rainhas não podem se atacar, explorando todas as possibilidades que isso acontece. 	A solução para tal problema está na disposição das mesmas, no tabuleiro, colocando de forma não paralela e alternada entre as pistas. Para tal solução a inteligência artificial teria que ser implementada com algoritmos de ordenação por espaço como o A* usando-se busca uniforme, para analisar os casos.


%-----------------------------------
\begin{figure}[h]
   \centering
   \includegraphics[width=4cm]{nrainhas1.png}
   \caption{Figura demostração do problema}
   \label{fig:nrainhas1} 
\end{figure}
%----------------------------------------

%_________
\subsection{Estado Inicial}
 O tabuleiro estará vazio, não tendo outras peças a não ser as rainhas.


%_________
\subsection{Ações}
 \textbf{ mover: em qualquer quantidade de casas nas direções} : Frente, trás, esquerda, direita, e diagonais.
%_________
\subsection{Modelo de Transição}
    Mover as peças, 1 turno por vez.

%_________
\subsection{Custo de Caminho}
  14 rainhas --> 365596 soluções
%_________
\subsection{Teste do Objetivo}
 Quando estiver a quantidade máxima de rainhas no tabuleiro, sem que elas se ataquem.

%###################

%-------------------
\section{Sudoku}
%-------------------

%_________
\subsection{Introdução}
	O Sudoku é um jogo relativamente simples, mas complexo similar a palavra cruzada, no qual deve-se colocar números de 1 – 9, em quadrados, de modo que não se repitam, na vertical e na horizontal, e dentro do próprio quadrado. É um bom exercício de lógica e de muita concentração, pois deve-se verificar essas 3 situações simultaneamente, o tabuleiro e os quadrantes já possuem um alguns números pré definidos, o jogador só tem que colocar os números respeitando essas regras, oque claramente demostra um problema de ordenação, aliado a busca uniforme, pois deve-se chegar de forma ordenada, tanto horizontal, quanto a vertical, e ainda o quadrante, oque demostra que é a melhor escolha, respeita a estrutura do jogo.
%_________
\subsection{Estado Inicial}
 Tabuleiro com alguns números pre estabelecidos.
%-----------------------------------
\begin{figure}[h]
   \centering
   \includegraphics[width=4cm]{sudoku.png}
   \caption{Figura demostração do problema}
   \label{fig:sudoku} 
\end{figure}
%----------------------------------------
 
%_________
\subsection{Ações}
 \textbf{ inserir números de 1 a 9}, Nas casas vazias, respeitando de modo que não se repitam, na vertical e na horizontal, e dentro do próprio quadrado.

%_________
\subsection{Modelo de Transição}
    \textbf{ checar não se repete}  Antes de colocar o número não está na horizontal, na vertical, e no quadrado

%_________
\subsection{Custo de Caminho}
    Dependendo da dificuldade até 56 movimentos.
%_________
\subsection{Teste do Objetivo}
Se a soma na horizontal der 45 e na horizontal der 45, e a soma dos números no quadrado der 45,  está correto. 


%###################

%----------------------------------
\section{Considerações Finais}
%----------------------------------
Este trabalho teve a pretensão de apresentar quatro problemas e analisar sua resolução do ponto de vista da inteligência artificial, perante a perspectiva simplória deste que vós escreve, mostrando que a computação clássica ainda é preza em suas bases, pois a computação quântica no qual esses problemas são muito mais simples ainda não se iniciou essa era, e que nós ainda não estamos prontos para tamanho avanço. 

Permanece perante todos o mistério que esses problemas, muitos outros sempre irão ter soluções mais eficientes, mas que estão limitadas a nosso conhecimento atual, visto que ainda a ciência cada dia mais evolui, mas nunca chegará a exatidão, pois essa, só pertence a fonte de tudo.  
%###################

%-------------------
\section{Referencias Blibiográficas}
%-----------------------------

%TÍTULO da matéria. Nome do site, ano. Disponível em: . Acesso em: dia, mês e ano.
CANIBAIS E MISSIONARIOS .Wikipedia. 2011. Disponivel em: 
\url{https://pt.wikipedia.org/wiki/Problema_dos_canibais_e_mission%C3%A1rios}



COMO SOLUCIONAR O CUBO MÀGICO. WikiHow. 2011. Disponivel em: 
\url{https://pt.wikihow.com/Resolver-um-Cubo-M%C3%A1gico}


O PROBLEMA DAS N-RAINHAS.  Ricardo Rocha. 2010. Disponivel em:
\url{https://www.dcc.fc.up.pt/~ricroc/aulas/0405/soII/prat/exerc2.html}


SUDOKU ONLINE . sudoku.com. 2020. Disponivel em:
\url{https://sudoku.com/br}


%-------------------------------------------
\end{document}
